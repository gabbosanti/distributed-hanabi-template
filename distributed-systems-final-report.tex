\documentclass{scrartcl}
\usepackage[utf8]{inputenc}
\usepackage{hyperref}
\usepackage{url}
\usepackage{natbib}
\usepackage{graphicx}
\usepackage{cleveref} % this must be the last package to be loaded

\newcommand{\emailaddr}[1]{\href{mailto:#1}{\texttt{#1}}}

\title{\LARGE
    Final Report : Distributed Hanabi
}

\subtitle{Final Report for the Distributed Systems Course 2025-2026}

\author{
    Blagoja Savevski \\ \emailaddr{blagoja.savevski@studio.unibo.it}
    \and
    Gabriele Santi \\ \emailaddr{gabriele.santi@studio.unibo.it}
    \and
    Pengyue Xu \\ \emailaddr{pengyue.xu@studio.unibo.it}
}

\date{\today}

\begin{document}

\maketitle

\begin{abstract}
Hanabi is a cooperative multiplayer card game for 2-5 players where players score by guessing their own hidden cards based on hints given by other players. This project aims to implement a working version of the game that manages the shared state consistently and reliably, while being able to tolerate disconnects from a client or the server. 
\end{abstract}

\section{Concept}\label{concept}

The aim of this project is to design and implement a distributed version of the game Hanabi.
It consist on a web-based application, where players can connect to a server and play togheter directly from their browsers.
A group between 2 and 5 players from distinct machines can play togheter and form a team, by starting the client and connecting to the central server.
The server is responsible for creating a new game and assigning a unique game ID, enabling persistent storage of its state and potential recovery in case of system restart.
Playes visualize the game shared state on their clients and, on each turn, send command to the server. Each command is validated by the server, according to Hanabi rules, then updates the game state accordingly and notifies all players of the new state.

Why is distribution needed?
  %
  \begin{itemize}
    \item geographically distributed environments?
    \item computation speedup?
    \item resource sharing?
    \item fault tolerance?
    \item other reasons?
  \end{itemize}

\section{Requirements Elicitation and Analysis}\label{requirements}

\begin{itemize}
  \item The requirements must explain \textbf{what} (not how) the software
  being produced should do.
  %
  \begin{itemize}
    \item you should not focus on the particular problems, but exclusively on
    what you want the application to do.
  \end{itemize}

  \item Requirements must be clearly identified, and possibly numbered

  \item Requirements are divided into:
  %
  \begin{itemize}
    \item \textbf{Functional}: some functionality the software should provide to the user

    \item \textbf{Non-functional}: requirements that do not directly concern behavioural aspects, such as consistency, availability, etc.
    \item \textbf{Implementation}: constrain the entire phase of system realization, 
    for instance by requiring the use of a specific programming language and/or a specific software tool
    %
    \begin{itemize}
      \item these constraints should be adequately justified by political / economic / administrative reasons\ldots{}
      \item \ldots{} otherwise, implementation choices should emerge \emph{as  a consequence of} design
    \end{itemize}
  \end{itemize}

  \item If there are domain-specific terms, these should be explained in a glossary
  
  \item Each requirement must have its own \textbf{acceptance criterion}
  %
  \begin{itemize}
    \item these will be important for the validation phase
  \end{itemize}
\end{itemize}

\subsection{Relevant Distributed System Features}
\label{ds-features}

Motivate which distributed system features are relevant for your project, and which are not.
%
\begin{itemize}
  \item transparency
  \begin{itemize}
    \item does your system need to hide distribution details from users or developers?
    \item is it important that failures, location, or replication are invisible?
  \end{itemize}

  \item fault tolerance, dependability -- availability, reliability, integrity, maintainability, safety
  \begin{itemize}
    \item what happens if a component fails? is uninterrupted service required?
    \item is data loss or corruption unacceptable?
    \item how quickly must the system recover from faults?
  \end{itemize}

  \item scalability
  \begin{itemize}
    \item will the system need to handle increasing numbers of users, requests, or data?
    \item is it expected to grow over time?
  \end{itemize}

  \item security, trust
  \begin{itemize}
    \item is sensitive data being processed or stored?
    \item are there multiple user roles with different permissions?
    \item is authentication or authorization required?
  \end{itemize}

  \item resource sharing
  \begin{itemize}
    \item do multiple users or components need access to shared resources?
    \item is coordination or synchronization needed?
  \end{itemize}

  \item openness, interoperability, heterogeneity of components
  \begin{itemize}
    \item Will your system interact with external systems or use components built with different technologies?
    \item Is standardization or compatibility important?
  \end{itemize}

  \item evolvability, maintainability
  \begin{itemize}
    \item Will the system need to be updated or extended after deployment?
    \item Is long-term maintenance a concern?
  \end{itemize}

  \item performance, concurrency, computation / communication efficiency, bandwidth
  \begin{itemize}
    \item Are there strict requirements on response time or throughput?
    \item Will many operations happen in parallel?
    \item Is network usage a concern?
  \end{itemize}

  \item economy, costs  
  \begin{itemize}
    \item Are there budget constraints for development, deployment, or operation?
    \item Is minimizing resource usage important?
  \end{itemize}
\end{itemize}

\section{Design}\label{design}

This chapter explains the strategies used to meet the requirements identified in the analysis. 
%
Ideally, the design should be the same, regardless of the technological choices made during the implementation phase.

\begin{quote}
You can re-arrange the sections as you prefer, but all the sections must be present in the end
\end{quote}

\noindent
\textbf{Important:} try to motivate your design choices in relation to the requirements and features identified 
in \cref{requirements} and \cref{ds-features}.

\subsection{Architecture}\label{architecture}

\begin{itemize}
  \item Which architectural style?
  %
  \begin{itemize}
    \item why?
  \end{itemize}
\end{itemize}

\subsection{Infrastructure}\label{infrastructure}

\begin{itemize}
  \item are there \emph{infrastructural components} that need to be introduced? \emph{how many}?
  %
  \begin{itemize}
    \item e.g.~\emph{clients}, \emph{servers}, \emph{load balancers},
    \emph{caches}, \emph{databases}, \emph{message brokers},
    \emph{queues}, \emph{workers}, \emph{proxies}, \emph{firewalls},
    \emph{CDNs}, \emph{etc.}
  \end{itemize}

  \item how do components \emph{distribute} over the network? \emph{where}?
  %
  \begin{itemize}
    \item e.g.~do servers / brokers / databases / etc. sit on the same
    machine? on the same network? on the same datacenter? on the same
    continent?
  \end{itemize}

  \item how do components \emph{find} each other?
  %
  \begin{itemize}
    \item how to \emph{name} components?
    \item e.g.~DNS, \emph{service discovery}, \emph{load balancing},
    \emph{etc.}
  \end{itemize}
\end{itemize}

\begin{quote}
Component diagrams are welcome here
\end{quote}

\subsection{Modelling}\label{modelling}

\begin{itemize}
  \item which \textbf{domain entities} are there?
  %
  \begin{itemize}
    \item e.g.~\emph{users}, \emph{products}, \emph{orders}, \emph{etc.}
  \end{itemize}
  
  \item how do \emph{domain entities} \textbf{map to} \emph{infrastructural components}?
  %
  \begin{itemize}
    \item e.g.~state of a video game on central server, while inputs/representations on clients
    \item e.g.~where to store messages in an IM app? for how long?
  \end{itemize}

  \item which \textbf{domain events} are there?
  %
  \begin{itemize}
    \item e.g.~\emph{user registered}, \emph{product added to cart}, \emph{order placed}, \emph{etc.}
  \end{itemize}
  
  \item which sorts of \textbf{messages} are exchanged?
  %
  \begin{itemize}
    \item e.g.~\emph{commands}, \emph{events}, \emph{queries}, \emph{etc.}
  \end{itemize}

  \item what information does the \textbf{state} of the system comprehend
  %
  \begin{itemize}
    \item e.g.~\emph{users' data}, \emph{products' data}, \emph{orders' data}, \emph{etc.}
  \end{itemize}
\end{itemize}

\begin{quote}
Class diagram are welcome here
\end{quote}

\subsection{Interaction}\label{interaction}

\begin{itemize}
  \item how do components \emph{communicate}? \emph{when}? \emph{what}?
  \item \emph{which} \textbf{interaction patterns} do they enact?
\end{itemize}

\begin{quote}
Sequence diagrams are welcome here
\end{quote}

\subsection{Behaviour}\label{behaviour}

\begin{itemize}
  \item how does \emph{each} component \textbf{behave} individually (e.g.~in \emph{response} to \emph{events} or messages)?
  \begin{itemize}
    \item some components may be \emph{stateful}, others \emph{stateless}
  \end{itemize}

  \item which components are in charge of updating the \textbf{state} of the system? \emph{when}? \emph{how}?
\end{itemize}

\begin{quote}
State diagrams are welcome here
\end{quote}

\subsection{Data and Consistency Issues}\label{data-and-consistency-issues}

\begin{itemize}
  \item Is there any data that needs to be stored?
  %
  \begin{itemize}
    \item \emph{what} data? \emph{where}? \emph{why}?
  \end{itemize}
  
  \item how should \emph{persistent data} be \textbf{stored}?
  %
  \begin{itemize}
    \item e.g.~relations, documents, key-value, graph, etc.
    \item why?
  \end{itemize}
  
  \item Which components perform queries on the database?
  %
  \begin{itemize}
    \item \emph{when}? \emph{which} queries? \emph{why}?
    \item concurrent read? concurrent write? why?
  \end{itemize}
  
  \item Is there any data that needs to be shared between components?
  %
  \begin{itemize}
    \item \emph{why}? \emph{what} data?
  \end{itemize}
\end{itemize}

\subsection{Fault-Tolerance}\label{fault-tolerance}

\begin{itemize}
  \item Is there any form of data \textbf{replication} / federation / sharing?
  %
  \begin{itemize}
    \item \emph{why}? \emph{how} does it work?
  \end{itemize}
  
  \item Is there any \textbf{heart-beating}, \textbf{timeout}, \textbf{retry mechanism}?
  %
  \begin{itemize}
    \item \emph{why}? \emph{among} which components? \emph{how} does it work?
  \end{itemize}
  
  \item Is there any form of \textbf{error handling}?
  %
  \begin{itemize}
    \item \emph{what} happens when a component fails? \emph{why}? \emph{how}?
  \end{itemize}
\end{itemize}

\subsection{Availability}\label{availability}

\begin{itemize}
  \item Is there any \textbf{caching} mechanism?
  %
  \begin{itemize}
    \item \emph{where}? \emph{why}?
  \end{itemize}
  
  \item Is there any form of \textbf{load balancing}?
  %
  \begin{itemize}
    \item \emph{where}? \emph{why}?
  \end{itemize}

  \item In case of \textbf{network partitioning}, how does the system behave?
  %
  \begin{itemize}
    \item \emph{why}? \emph{how}?
  \end{itemize}
\end{itemize}

\subsection{Security}\label{security}

\begin{itemize}
  \item Is there any form of \textbf{authentication}?
  %
  \begin{itemize}
    \item \emph{where}? \emph{why}?
  \end{itemize}

  \item Is there any form of \textbf{authorization}?
  %
  \begin{itemize}
    \item which sort of \emph{access control}?
    \item which sorts of users / \emph{roles}? which \emph{access rights}?
  \end{itemize}

  \item Are \textbf{cryptographic schemas} being used?
  %
  \begin{itemize}
    \item e.g.~token verification,
    \item e.g.~data encryption, etc.
  \end{itemize}
\end{itemize}

\section{Implementation}\label{implementation}

Please report here all the implementation (technology-dependent) choices you made while implementing your design.

\begin{quote}
  If you run out of time for this project, you may consider leaving some aspect unimplemented, 
  and simply discuss how you would have implemented them.
  %
  In this case, better would be to discuss unimplemented features in \cref{future-works}.
\end{quote}

\begin{itemize}
  \item which \textbf{network protocols} to use?
  %
  \begin{itemize}
    \item e.g.~UDP, TCP, HTTP, WebSockets, gRPC, XMPP, AMQP, MQTT, etc.
  \end{itemize}
  
  \item how should \emph{in-transit data} be \textbf{represented}?
  %
  \begin{itemize}
    \item e.g.~JSON, XML, YAML, Protocol Buffers, etc.
  \end{itemize}
  
  \item how should \emph{databases} be \textbf{queried}?
  %
  \begin{itemize}
    \item e.g.~SQL, NoSQL, etc.
  \end{itemize}
  
  \item how should components be \emph{authenticated}?
  %
  \begin{itemize}
    \item e.g.~OAuth, JWT, etc.
  \end{itemize}

  \item how should components be \emph{authorized}?
  %
  \begin{itemize}
    \item e.g.~RBAC, ABAC, etc.
  \end{itemize}
\end{itemize}

\subsection{Technological details}\label{technological-details}

\begin{itemize}
  \item any particular \emph{framework} / \emph{technology} being exploited goes here
\end{itemize}

\section{Validation}\label{validation}

\subsection{Automatic Testing}\label{automatic-testing}

\begin{itemize}
  \item how were individual components \textbf{\emph{unit}-test}ed?
  
  \item how was communication, interaction, and/or integration among components tested?
  
  \item how to \textbf{\emph{end-to-end}-test} the system?
  %
  \begin{itemize}
    \item e.g.~production vs.~test environment
  \end{itemize}

  \item for each test specify:
  %
  \begin{itemize}
    \item rationale of individual tests
    \item how were the test automated
    \item how to run them
    \item which requirement they are testing, if any
  \end{itemize}
\end{itemize}

\begin{quote}
recall that \emph{deployment} \textbf{automation} is commonly used to \emph{test} the system in \emph{production-like} environment
\end{quote}

\begin{quote}
recall to test corner cases (crashes, errors, etc.)
\end{quote}

\subsection{Acceptance test}\label{acceptance-test}

\begin{itemize}
  \item did you perform any \emph{manual} testing?
  %
  \begin{itemize}
    \item what did you test?
    \item why wasn't it automatic?
  \end{itemize}
\end{itemize}

\section{Deployment}\label{deployment}

\begin{itemize}
  \item should one install your software from scratch, how to do it?
  %
  \begin{itemize}
    \item provide instructions
    \item provide expected outcomes
  \end{itemize}

  \item what software should be installed on the machines to run your project? which versions?
  
  \item should one set environment variables or configuration files?
  
  \item if you're using containerization (e.g.~Docker), 
  describe how you engineered your deployment scrips (e.g. \texttt{docker-compose.yml} files)
\end{itemize}

\section{User Guide}\label{user-guide}

\begin{itemize}
  \item how to use your software?
  %
  \begin{itemize}
    \item provide instructions
    \item provide expected outcomes
    \item provide screenshots if possible
  \end{itemize}
\end{itemize}

\section{Self-evaluation}\label{self-evaluation}

\begin{itemize}
  \item An individual section is required for each member of the group
  \item Each member must self-evaluate their work, listing the strengths and weaknesses of the product
  \item Each member must describe their role within the group as objectively as possible.
\end{itemize}

It should be noted that each student is only responsible for their own section.

\section{Future works}\label{future-works}

Discuss possible future works, improvements, extensions, optimizations, etc.

Also discuss here any unimplemented feature, and how you would have implemented it.

\bibliographystyle{plain}
\bibliography{references}

\end{document}
